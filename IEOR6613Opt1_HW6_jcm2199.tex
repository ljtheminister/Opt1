\documentclass{article}

\usepackage[margin=1.0in]{geometry}
\usepackage{amssymb, amsmath, parallel, mathtools, graphicx, array}

\begin{document}

\title{IEOR 6613 - Optimization I\\ HW 6:  6.6, 6.8, 7.1, 7.6, 7.13, 7.26}

\author{John Min\\ jcm2199}
\date{November 20, 2013}
\maketitle

\section*{6.6}
Consider a linear programming problem in standard form in which the matrix $\mathbf{A}$ has the following structure: \\
$$ \mathbf{A = } 
\begin{bmatrix} \mathbf{A_{00}} & \mathbf{A_{01}} & \cdots & \cdots & \mathbf{A_{0n}} \\
				\mathbf{A_{10}} & \mathbf{A_{11}} & \cdots & \cdots & \cdots	\\
				\vdots & & \mathbf{A_{22}} \\
				\vdots & & & \ddots \\
				\mathbf{A_{n0}} & & & & \mathbf{A_{nn}}
\end{bmatrix} $$. \\
(All submatrices other than those indicated are zero.)  Show how a decomposition method can be applied to a problem with this structure.  Do not provide details, as long as you clearly indicate the master probelm and the subproblems.  \emph{Hint:} Decompose twice. \\

\noindent


\section*{6.8}  
Consider the Dantzig-Wolfe decomposition method and suppose that we are at a basic feasibile solution to the master problem. \\

\noindent
\textbf{(a)} Show that at least one of the variables $\lambda_1^j$ must be a basic variable. \\

\noindent
\textbf{(b)}  Let $r_1$ be the current value of the simplex multipler associated with the first convexity constraint (6.12), and let $z_1$ be the optimal cost in the first subproblem.  Show that $z_1 \leq r_1$.  

\section*{7.1 (The caterer problem)}
A catering company must provide to a client $r_i$ tablecloths on each of $N$ consecutive days.  The catering company can buy new tablecloths at a price of $p$ dollars each, or launder the used ones.  Laundering can be done at a fast service facility that makes the tablecloths unavailable for the nexet $n$ days and costs $f$ dollars per tablecloth, or at a slower facility that makes tablecloths unavailable for the next $m$ days (with $m > n$) at a cost of $g$ dollars per tablecloth ($g < f$).  The caterer's problem is to decide how to meet the  client's demand at minimum cost, starting with no tablecloths and under the assumption that any leftover tablecloths have no value. \\

\noindent \textbf{(a)}
Show that the problem can be formulated as a network flow problem.  \emph{Hint:}  Use a node corresponding to clean tablecloths and a node corresponding to dirty tablecloths for each day; more nodes may also be needed. \\

\noindent \textbf{(b)}
Show explicitly the form of hte network if $N = 5, n=1, m=3$.  


\end{document}